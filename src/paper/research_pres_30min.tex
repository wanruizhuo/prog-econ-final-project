\documentclass[11pt]{beamer}
% \documentclass[11pt,handout]{beamer}
\usepackage[T1]{fontenc}
\usepackage[utf8]{inputenc}
\usepackage{float, afterpage, rotating, graphicx}
\usepackage{epstopdf}
\usepackage{longtable, booktabs, tabularx}
\usepackage{fancyvrb, moreverb, relsize}
\usepackage{eurosym, calc}
\usepackage{amsmath, amssymb, amsfonts, amsthm, bm} 


\usepackage[
    natbib=true,
    bibencoding=inputenc,
    bibstyle=authoryear-ibid,
    citestyle=authoryear-comp,
    maxcitenames=3,
    maxbibnames=10,
    useprefix=false,
    sortcites=true,
    backend=biber
]{biblatex}
\AtBeginDocument{\toggletrue{blx@useprefix}}
\AtBeginBibliography{\togglefalse{blx@useprefix}}
\setlength{\bibitemsep}{1.5ex}
\addbibresource{refs.bib}



\hypersetup{colorlinks=true, linkcolor=black, anchorcolor=black, citecolor=black, filecolor=black, menucolor=black, runcolor=black, urlcolor=black}

\setbeamertemplate{footline}[frame number]
\setbeamertemplate{navigation symbols}{}
\setbeamertemplate{frametitle}{\centering\vspace{1ex}\insertframetitle\par}


\begin{document}

\title{Panel Data Regression Model Estimation and Inference}

\author[Ruizhuo Wan]
{
{\bf Ruizhuo Wan}\\
{\small University of Bonn}\\[1ex]
}


\begin{frame}
    \titlepage
    \note{~}
\end{frame}


\begin{frame}
Introduction

Panel data can be seen as the pool of cross-sectional units observed over several time periods. It provides more informative data, controls individual heterogeneity and displays dynamics of change.
\end{frame}

\begin{frame}
General Model
\begin{align*}
    y_{it}=\alpha+X_{it}'\beta+u_{it} \;\; with \;\; i=1,...,N; \; t=1,...,T
\end{align*}

\begin{align*}
    u_{it}=\mu_i+v_{it}
\end{align*}
\begin{align*}
    y=\alpha\iota_{NT}+X\beta+u=Z\delta+u
\end{align*}
\begin{align*}
    u=Z_{\mu}\mu+v
\end{align*}
\end{frame}

\begin{frame}
Fixed Effects Model
\begin{itemize}
    \item $\mu_i$ are unknown fixed parameters for each individual,
    \item $v_{it}$ are $IID(0, \sigma_v^2)$ with finite $\sigma_v^2$,
    \item $X_{it}$ are independent of $v_{it}$ $\forall \; i, t$.
\end{itemize}
After premuliplication,
\begin{align*}
    Qy=Q\alpha\iota_{NT}+QX\beta+QZ_{\mu}\mu+Qv=QX\beta+Qv
\end{align*}

\end{frame}

\begin{frame}
Estimation

Applying OLS leads to the within-estimator for $\beta$.
OLS minimizes the sum of squared residuals with respect to $\widehat{\beta}$:
\begin{eqnarray}
    \min_{\widehat{\beta}}\;e'e&=&(Qy-QX\widehat{\beta})'(Qy-QX\widehat{\beta})\nonumber\\&=&y'Qy-\widehat{\beta}QX'y-y'QX\widehat{\beta}+\widehat{\beta}X'QX\widehat{\beta}\nonumber\\&=&y'Qy-2\widehat{\beta}X'Qy+\widehat{\beta}'X'QX\widehat{\beta}
\end{eqnarray}
Derivation leads to FOC:
\begin{equation}
    \frac{\partial e'e}{\partial \widehat{\beta}}=-2X'Qy+2X'QX\widehat{\beta}\overset{!}{=}0
\end{equation}
The OLS estimator is:
\begin{equation}
    \widehat{\beta}=(X'QX)^{-1}X'Qy
\end{equation}
with $var(\widehat{\beta})=\sigma_v^2(X'QX)^{-1}$.
\end{frame}

\begin{frame}
Statistical Inference

\begin{itemize}
    \item \textit{(A1) Linearity}: The model is linear in parameters $\alpha$, $\beta$, $\mu$ and error $v$
    \item \textit{(A2) Independence}: $\{ X_i, y_i \}_{i=1}^N \; i.i.d.$ (Random Sampling in cross-section)
    \item \textit{(A3) Strict Exogeneity}: $\mathop{\mathbb{E}}[u_{it}|X_i, \mu_i]=0$
    \item \textit{(A4) Identifiability}: $rank(\Tilde{X})=K<NT$ and $\mathop{\mathbb{E}}[\Tilde{x}_i'\Tilde{x}_i]$ is positive definite and finite
\end{itemize}
If the errors are homoskedastic with no serial correlation, the variance of the FE estimator $\hat{\beta}$ can be estimated.
\begin{itemize}
    \item \textit{(A5) Homeskedasticity, no serial correlation}: $V[u_i|X_i, \mu_i]=\sigma_u^2I,\,\sigma_u^2>0$ and finite
\end{itemize}
\begin{equation}
    \widehat{Var}(\widehat{\beta}|X)=\widehat{\sigma}_u^2(\Tilde{X}'\Tilde{X})^{-1}
\end{equation}
\end{frame}

\begin{frame}
Random Effects Model

\begin{itemize}
    \item $\mu_i$ is assumed to be random
    \item $\mu_i$ and $\nu_{it}$ are independent identically distributed with zero means. In the meanwhile, the variance of $\mu_i$ is $\sigma_\mu ^2$ and the variance of $\nu_{it}$ is $\sigma_\nu ^2$
    \item $X_{it}$ are all independent of $\mu_i$ and $\nu_{it}$ for all i and t
\end{itemize}
\end{frame}

\begin{frame}
Estimation
   Variance-Covariance Matrix

    \begin{equation}
    \begin{split}
    \Omega &= E(uu') = Z_\mu E(\mu \mu')Z_\mu' + E(\nu\nu')\\
    &= \sigma_\mu^2(I_N \otimes J_T) + \sigma_\nu^2(I_N \otimes I_T)\\
    &= \sigma_1^2P + \sigma_\mu^2Q
    \end{split}
    \end{equation}
    \begin{equation}
    \sigma_1^2 = T\sigma_\mu^2 + \sigma_\nu^2
    \end{equation}
    \begin{equation}
    \Omega^{-1} = \frac{1}{\sigma_1^2} P + \frac{1}{\sigma_\nu^2} Q
    \end{equation}
\end{frame}

\begin{frame}
\begin{itemize}
    \item Within Regression
        \begin{equation}
        y_{it} - \bar y_{i.} = \beta(x_{it} - \bar x_{i.}) + (\nu_{it} - \bar\nu_{i.})
        \end{equation}
        \begin{equation}
        {\widehat{\widehat{\sigma}}}_\nu^2 = [y'Qy - y'QX(X'QX)^{-1}X'Qy]/[N(T - 1) - K]
        \end{equation}
    \item Between Regression
        \begin{equation}
        \bar y_{i.} = \alpha + \bar X'_{i.}\beta + \bar \mu_{i.} \ \ \ \ i = 1, \ldots, N
        \end{equation}
        \begin{equation}
        \sqrt T \bar y_{i.} = \alpha \sqrt T + \sqrt T \bar X'_{i.} \beta + \sqrt T \bar u_{i.}
        \end{equation}
        \begin{equation}
        {\widehat{\widehat{\sigma}}}_1^2 = (y'Py - y'PZ(Z'PZ)^{-1}Z'Py)/(N - K -1)
        \end{equation}
    \item GLS Estimator
        \begin{equation}
        \widehat\beta_{GLS} = [(X'QX/\sigma_\nu^2) + X'(P - \bar J_{NT})X/\sigma_1^2]^{-1}[(X'Qy/\sigma_\nu^2) + X'(P - \bar J_{NT})y/\sigma_1^2]
        \end{equation}
        \begin{equation}
        \widehat\beta_{GLS} = [W_{XX} + \phi^2B_{XX}]^{-1}[W_{Xy} + \phi^2B_{Xy}]
        \end{equation}
    where $W_{XX} = X'QX$, $\phi^2 = \sigma_\nu^2 / \sigma_1^2$, $B_{XX} = X'(P - \bar J_{NT})X$, $W_{Xy} = X'Qy$ and $B_{Xy} = X'(P - \bar J_{NT})y$
\end{itemize}
\end{frame}

\begin{frame}
Statistical Inference
Estimator is unbiased under the assumptions of:
\begin{itemize}
    \item \textit {(A1) - (A4)}
    \item \textit{(A6) Identically Independently Distributed Error Components (as states in the part of assumptions)}
    \item if $\sigma_\mu^2 = 0$, then $\phi^2 = 1$ so that the random effects estimator ($\widehat \beta_{GLS}$) goes to OLS estimator ($\widehat \beta_{OLS}$)
    \item if the number of time periods converges to infinite, then $\phi^2$ will converge to zero, which leads $\widehat \beta_{GLS}$ tending to $\widetilde \beta_{Within}$, the fixed effect estimator
    \item if $W_{XX}$ is pretty large when comparing to $B_{XX}$, then $\widehat \beta_{GLS}$ will also goes to $\widetilde \beta_{Within}$
    \item if on the contratory, $B_{XX}$ is pretty large comparing to $W_{XX}$, $\widehat \beta_{GLS}$ tends to $\widehat \beta_{Between}$
    \item \textit {(A7) Homoskedasticity and Serial Correlation}:
    \begin{equation}
    var(u_{it}) = \sigma_\mu^2 + \sigma_\nu^2 \text{, for any i and t}
    \end{equation}
\end{itemize}
\end{frame}

\begin{frame}
Hausman Specification Test
Test Statistics
    \begin{equation}
    \widehat q_1 = \widehat \beta_{GLS} - \widetilde \beta_{Within}
    \end{equation}
    \begin{equation}
    m_1 = \widehat q_1'[var(\widehat q_1)]^{-1}\widehat q_1
    \end{equation}
\end{frame}


% Print black screen only in presentation mode for finishing up.
\mode<beamer> {
    \beamersetaveragebackground{black}
    \begin{frame}
        \frametitle{}
    \end{frame}

    \beamersetaveragebackground{white}
}

\begin{frame}[allowframebreaks]
    \frametitle{References}
    
    \renewcommand{\bibfont}{\normalfont\footnotesize}
    \printbibliography
    
    
\end{frame}

\end{document}
